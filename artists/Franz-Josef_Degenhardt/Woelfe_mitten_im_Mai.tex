\subsection*{Wölfe mitten im Mai\index{Wölfe mitten im Mai}   \hrulefill { \small Franz-Josef Degenhardt}}
\addcontentsline{toc}{subsection}{Wölfe mitten im Mai}
\begin{guitar}
  \begin{multicols}{2}
A[Dm]ugust der Schäfer hat Wölfe gehört,
Wö[B]lfe m[C]itten im M[F]ai, zwar nur z[C]wei,
aber A[Dm]ugust der schwört,
s[Dm]ie hätten zusammen das Fraßlied geheult,
d[B]as aus frü[C]herer Z[F]eit, und er schr[C]eit.
und sein H[Dm]ut ist verbeult.
Schr[F]eit: ''Rasch, holt die Sensen sonst i[C]st es zu spät.
Schlagt sie t[B]ot, noch eher der H[C]ahn dreimal krä[F]ht.''
Doch wer hö[Dm]rt schon auf einen [C]alten Hut
und ist auf der H[B]ut? Und ist auf der [Dm]Hut?



August der Schäfer ward niemehr geseh'n,
nur sein alter Hut, voller Blut,
schwamm im Bach. Circa zehn
hat dann später das Dorfhexenkind
nachts im Steinbruch entdeckt, blutbefleckt
und die Schnauze im Wind.
Dem Kind hat die Mutter den Mund zugehext,
hat geflüstert: ''Bist still oder du verreckst!
Wer den bösen Wolf nicht vergißt, mein Kind,
bleibt immer ein Kind. Bleibt immer ein Kind.''


Schon schnappten die Hunde den Wind, und im Hag
rochen Rosen nach Aas. Kein Schwein fraß.
Eulen jagten am Tag.
Hühner verscharrten die Eier im Sand.
Speck im Fang wurde weich. Aus dem Teich
krochen Karpfen an Land.
Da haben die Greise zahnlos gelacht
gezischelt: ``Wir haben's gleich gesagt.
Düngt die Felder wieder mit altem Mist,
sonst ist alles Mist - sonst ist alles Mist.''

Dann, zu Johannis, beim Feuertanz
-keiner weiß heut mehr wie - 
waren sie
plötzlich da. Aus Geäst
sprangen sie in den Tanzkreis; zu schnell
bissen Bräute ins Gras,
und zu blaß
schien der Mond; aber hell,
hell brannte Feuer aus trockenem Moos,
brannte der wald bis hinunter zum Fluß.
``Kinder, spielt, vom Rauch dort wissen wir nichts
und riechen auch nichts - und riechen auch nichts.''


\columnbreak
``Jetzt kommen Zeiten, da heißt es, heraus
mit dem Gold aus dem Mund.
Seid klug und
wühlt euch Gräben ums Haus.
Gebt eure Töchter dem rohesten Knecht,
jenem, der auch zur Not
nicht nur Brot
mit den Zähnen aufbricht.''
So sang der verschmuddelte Bauchladenmann
und pries Amulette aus Wolfszähnen an.
''Wickelt Stroh und Stacheldraht um den Hals
und haltet den Hals. Und haltet den Hals.''

Was ist dann doch in den Häusern passiert?
Bisse in Balken und Bett. Welches Fett
hat den Rauchfang verschmiert?
Wer gab den Wölfen die Kreide, das Mehl,
stäubte die Pfoten weiß? Welcher Greiß
glich dem Ziegengebell?
Und hat sich ein siebentes Geißlein versteckt?
Wurden Wackersteine im Brunnen entdeckt?
Viele Fragen, die nur einer hören will,
der stören will. Der stören will.

Doch jener Knecht mit dem Wildschweingebrech
- heute ein Touristenziel - weiß, wieviel
da geschah. Aber frech
hockt er im Käfig, frißt Blutwurst und lacht
wennn man ihn fragt. Und nur Schlag Null Uhr
zur Johannisnacht,
wenn von den Bergen das Feuerrad springt,
die Touristenschar fröhlich das Fraßlied singt,
beißt er wild ins Gitter, schreit: ''Schluß mit dem Lied!
's ein garstig Lied. 's ein garstig Lied.''

August der Schäfer hat Wölfe gehört,
Wölfe mitten im Mai, mehr als zwei.
Und der Schäfer, der schwört,
sie hätten zusammen das Fraßlied geheult,
das aus früherer Zeit, und er schreit.
Und sein Hut ist verbeult.
Schreit: ''Rasch, holt die Sensen sonst ist es zu spät.
Schlagt sie tot, noch ehe der Hahn dreimal kräht.''
Doch wer hört schon auf einen alten Hut
und ist auf der Hut? Und ist auf der Hut? 
\end{multicols}
\end{guitar}
