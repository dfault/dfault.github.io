\subsection*{Er muss uns jetzt was singen\index{Er muss uns jetzt was singen}   \hrulefill { \small Joint Venture}}
\addcontentsline{toc}{subsection}{Er muss uns jetzt was singen}
\begin{guitar}

[Dm]Vorne da, der [Fmaj7]Liedersänger, 
[C]nicht grade ein [G]Rattenfänger, 
buhlt beflissen um die Gunst 
des Pöbels hier für seine Kunst. 
Irgendwie wirkt das verkrampft, 
wie der da so jault und klampft. 
Als obs ihn alles sehr berührt. 
Man wird beschissen amüsiert. 

Bei allm, was ich ertragen hab, 
der Kerl, der schießt den Vogel ab, 
Der Traum vom großen Kollektiv, 
das ist so abgeschmackt naiv 
und so beschissen selbstgerecht, 
der ist so unglaublich schlecht. 
Schaut betroffen in die Runde 
und kommt zur ernsten Viertelstunde. 


[Dm]Wahrsch[Fmaj7]einlich hat er [C]Hasch geraucht. 
[Dm]Jetzt [Fmaj7]denkt er, daß die [C]Welt ihn braucht. 
[Dm]Kein[Fmaj7]em wirds was [C]bringen, 
[G]er muß uns jetzt was singen. 


Ja, der ist genau mein Typ, 
der hat die Welt so furchtbar lieb, 
daß er sich einfach besser fühlt, 
wenn er uns was mit Botschaft spielt. 
Alle nur zu amüsieren, 
heißt, sie hinters Licht zu führen, 
Schaut ergriffen in die Menge 
und dann zehn Minuten Länge. 


Wahrscheinlich hat er Hasch geraucht. 
Jetzt denkt er, daß die Welt ihn braucht. 
Keinem wirds was bringen, 
er muß uns jetzt was singen. 


Er singt vom Hungern und vom Friern, 
von Blumen, die ein Grab verziern. 
Am Ende kommt heraus, daß man 
in Kriegen niemals siegen kann. 
Daß alle Menschen Brüder sind, 
und die Antwort weiß der Wind. 
Ich hätt mich sicher für den Frieden 
auch ohne seinen Song entschieden. 


Wahrscheinlich hat er Hasch geraucht. 
Jetzt denkt er, daß die Welt ihn braucht. 
Keinem wirds was bringen, 
er muß uns jetzt was singen. 
\end{guitar}
